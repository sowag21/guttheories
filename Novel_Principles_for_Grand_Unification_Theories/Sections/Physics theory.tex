\section{Effective potential}
The effective potential is a function which to lowest order in pertubation theory is equal to the classical potential energy and to higher orders would include qunatum corrections \cite{Peskin:1995ev}. These corrections are generally not finite and owuld require renormalisation. The effective potential can be used for studying spontaneous symmetry breaking, which is based on minimizing the classical potential $V(\phi)$. However, the vacuum at a qunatum level isn't static, but rather fluctuating. Thus, using the effective potetial these qunatum fluctuation is taken intro account \cite{zee}.
To define the effective potential, we first need to define the effective action. To do this, we consider a scalar field theory $\phi$  with energy as the connected generating functional $E[J]$
\begin{equation}
    Z[J]=e^{-i E[J]}= \int \mathcal{D}\phi\,\phi \exp\left[i\int d^4x (\mathcal{L}[\phi]+J\phi)\right]\;,
\end{equation}
where both the external source $J$ and the field $\phi$ depends on the spacetime coordinate $x$. The equation is in the functional integral formalism. If we translate it to the cannonical formalism we recognize that the right hand side is simply an amplitud on the form
\begin{equation}
    \langle \Omega | \exp\left[-iHT\right] |\Omega\rangle\;,
\end{equation}
where $\ket \Omega$ is the vacuum state of the interacting Hamiltonian $H$, and $T$ is the relevant time interval. Thus, it is simply a vacuum to vacuum amplitude, in the presence of $J$. So $E[J]$ must represent the vacuum energy \cite{Peskin:1995ev}. To define an effective action, we first need to compute the expectation value of the field. This holds information on the state of the vacuum and the systems response to external perturbations
\begin{equation}    
    \phi_{c}(x)\equiv\frac{\delta E[J]}{\delta J(x)}\;,
\end{equation}
the expectation value $\phi_c(x)$ in the canonical formalism $\langle \Omega | \phi(x) |\Omega\rangle_J   $ is a weighted average of the fluctuations of the vacuum configuration and is sometimes called the classical field \cite{Peskin:1995ev}. Now evaluating the functional derivative
\begin{equation}
    \phi_c(x)=-\frac{1}{Z[J]}\int \mathcal{D}\phi\,\phi(x) \exp\left[i\int d^4x (\mathcal{L}[\phi]+J\phi)\right]=\langle \Omega | \phi(x) |\Omega\rangle_J\;.
\end{equation}
Now we can perform a Legendre transformation. This is done analogous to statistical mechanics, where the Legendre transformation relates free energy to the energy: $F = E-TS$, where the free energy is a function of temperature $T$ and the energy is a function of entropy $S$. In this case we have a function $E$ of $J$ and we transform it to the effective action $\Gamma$ which is a function of $\phi_c$\footnote{The difference in parenteses used $()$ and $[]$ refers to the function of or the functional of.}
\begin{equation}
    \Gamma[\phi_c]= -E[J] - \int d^4y\, J(y)\phi_c(y)\;.
    \label{eq:eff pot classical field}
\end{equation}
To better understand the role of the effective action, we take the functional derivative of it with respect to the classical field. This is analogue to the classic action $S[\phi]=\int \text{d}^4 x\, \mathcal{L}(\phi,\partial_\mu\phi)$ where varying the action with respect to the field $\phi(x)\to\phi(x) +\delta \phi(x)$ gives the equations of motion
\begin{equation}
    \frac{\delta S[\phi]}{\delta \phi}=0 \;.
\end{equation}
Now let us do the same for the quantum corrected action
\begin{equation}
    \frac{\delta \Gamma[\phi_c]}{\delta \phi_c(x)}=- \frac{\delta E[J]}{\delta \phi_c(x)} - \int d^4y\,  J(y)\frac{\delta \phi_c(y)}{\delta \phi_c(x)} - \int d^4y\,  \frac{\delta J(y)}{\delta \phi_c(x)}\phi_c(y)\;,
    \label{eq:def of eff pot}
\end{equation}
using the definition of the functional derivative \ref{conventions section} and that $E[J]$ depends on the source, we can rewrite the functional derivative in the first term using the chain rule
\footnote{The chain rule for functional derivatives is a generalization of the standard differentiation chain rule $\frac{\partial f}{\partial x}=\frac{\partial f}{\partial g}frac{\partial g}{\partial x}$ for a composotite function $f(g(x))$. For a functional derivative the difference is that $g(x)\to g_i (\vec x)$. This means that $f$ is a functional of $g$ which is a function of $x$, this is why we consider the continium limit and we have 
\begin{equation*}   
    \frac{\delta f[g]}{\delta g(x)}=\int  \frac{\delta f[g]}{\delta g(x)}\frac{\delta g(x)}{\delta x(y)}
\end{equation*}  
\cite{% https://wiki.physics.udel.edu/wiki_qttg/images/5/53/BOOK%3Ddft_an_advanced_course.pdf p. 411 and https://math.stackexchange.com/questions/235769/is-there-a-chain-rule-for-functional-derivatives 
}
\comment{rewrite this is not super correct} }  
\begin{equation}
    \frac{\delta \Gamma[\phi_c]}{\delta \phi_c(x)}=- \int d^4y\, \frac{\delta J(y)}{\delta \phi_c(x)} \frac{\delta E[J]}{\delta J(y)} - J(x) - \int d^4y\,  \frac{\delta J(y)}{\delta \phi_c(x)}\phi_c(y)\;,
\end{equation}
using the definition of the classical field in Equation \ref{eq:eff pot classical field}, we obtain
\begin{equation}
    \frac{\delta \Gamma[\phi_c]}{\delta \phi_c(x)}=- \int d^4y\, \frac{\delta J(y)}{\delta \phi_c(x)}(-\phi_x(y)) - J(x) - \int d^4y\,  \frac{\delta J(y)}{\delta \phi_c(x)}\phi_c(y)=- J(x)\;.
\end{equation}
Thus, with no external source $J$, the effective action has the same form as the classic action, when computing the equations of motion. The equation can therefore be interpreted as the quantum equation of motion. Using that the effective action, actually is an action we can expand the functional derivative $\Gamma[\phi_c]$
\begin{equation}
    \Gamma[\phi_c] = \int \text{d}^4 x\, \mathcal{L}_{\text{eff}}(\phi_c,\partial_\mu\phi_c)= \int \text{d}^4 x\, \left[-V_{\text{eff}}(\phi_c) + Z(\phi_c)(\partial_\mu \phi_c)^2 + \ddots \right]\;,
\end{equation}
where the $\mathcal{L}_{\text{eff}}$ is the effective lagrangian, $V_{\text{eff}}(\phi_c) $ is the effective potential, and the dots indicate higher powers of derivatives. Using this expansion of the effective action, the functional derivative can aslo be written as
\begin{equation}
    \frac{\delta \Gamma[\phi_c]}{\delta \phi_c(x)}=- \frac{\delta V_{\text{eff}}(\phi_c) }{\delta \phi_c(x)} frac{\delta }{\delta \phi_c(x)}\int \text{d}^4 x\, \left[ Z(\phi_c)(\partial_\mu \phi_c)^2 + \ddots \right]=- \frac{\delta V_{\text{eff}}(\phi_c) }{\delta \phi_c(x)} \;,
\end{equation}
for $\phi_c(x)$ constant in x, and thereby $J$ independent of $x$. Thus, we can write the effective potential as

\begin{equation}
    \frac{\delta V_{\text{eff}}(\phi_c) }{\delta \phi_c(x)} = J(x) \;.
\end{equation}
For the case of no source, the condition that the effective action has an extrenum is converted into a condition on the effective potential to have a minimum \comment{is minimimum correct here? it is not just an extrenum}
\begin{equation}
    \frac{\delta V_{\text{eff}}(\phi_c) }{\delta \phi_c(x)} = 0 \;.
\end{equation}
Consider Equation \ref{eq:def of eff pot} with no source term the vacuum expectation value is determined by minimizing $V_{\text{eff}}$ \cite{zee}.
\comment{I think I need an example of how to calculate it}



\section{The renormalisation group}



\subsection{The Callan-Symanzik Equation}

\subsection{The beta function and anomalous dimensions}

\subsubsection{An example: Non-Abelian Lagrangian}

\subsection{Fixed point analysis}

\subsubsection{Banks-Zacks fixed point}

So the result is
\begin{equation}
    \beta_{g_{1-loop}}=-\alpha_g^2 b_0 =\frac{-g^3}{16\pi^2}\left[ \frac{11}{3}C_2(H)-\frac{4}{3}n_fC(R)\right]\;.\label{eq:betafunction YM 1loop}
\end{equation}


\section{Chiral theories}
We want to consider chiral theories which are completely asymptotically free. Before doing that, we'll briefly consider what a chiral theory is. Chirality comes from the Dirac equation 
\begin{equation}
    (i\gamma^\mu \partial_\mu - m)\psi (x) =0\;,
\end{equation}
where $\gamma^\mu$ is the gamma matricies. \comment{I use the chiral rep / weyl rep} Rewriting the equation using the momentum operator $\slashed{p}=\gamma^\mu(i\partial_\mu)$ and the $2\times 2$ form of the gamma matricies
\begin{equation}
\left[ 
\begin{pmatrix}
0 & p^0 \\
p^0 & 0
\end{pmatrix}
- 
\begin{pmatrix}
0 & \vec\sigma \cdot \vec p \\
-\vec \sigma \cdot\vec{p}  & 0
\end{pmatrix}
- 
\begin{pmatrix}
m & 0 \\
0 & m
\end{pmatrix}
\right]
\begin{pmatrix}
\psi_L \\
\psi_R
\end{pmatrix}
= 0\;.
\end{equation}
From this it can be seen that massless particles come in two types, left- and right-handed and that they do not mix \cite{for dummies}. In this case an additional symmetry also applies for the field, namely the chiral symmetry $\psi\to e^{i\gamma^5\phi}\psi$ \cite{zee}. Chiral fermions are fermions which transform differently under the gauge group. This also means that you cannot combine left- and right-handed fermions and form a gauge invariant mass term in the lagrangian. Thus, in chiral theories fermions obtain mass at spontantous symmetry breaking like the Higgs mechanism \cite{Peskin:1995ev}.

To investigate chiral theories and how the gauge coupling in the theory change with energy scale, we want to consider the beta function of the lagrangian 
\begin{equation}
    \mathcal{L}=-\frac{1}{4} F^a_{\mu\nu}F^{a\mu\nu} + iT_{ij}\sigma^\mu D_\mu \bar T^{ij} + i\tilde{F}_k\sigma^\mu D_\mu \bar{\tilde{F}}^k+iF_m \sigma^\mu D_\mu \bar{F}^m\;, \label{eq:lagrangian YM Weyl}
\end{equation}
with Weyl fermions in fundamental $F$, anti-fundamental $\tilde{F}$, and two-index symmetric and antisymmetric representations. The lagrangian is thus a Yang-Mills gauge theory with Weyl fermions in different representations. The flavor indices on the anti-fundamental and the fundamental representation is $k=1,2,\ldots,(N\pm 4+p)$ and $m=1,2,\ldots,p$. The interval of these indicies will make sense later. The Weyl fermion representations in the lagrangian are from two chiral theory models. The Georgi-Glashow (anti-symmetric two-index representation) and the Bars-Yankielowicz (symmetric two-index representation) models \cite{M_lgaard_2017}. 

The beta function for Yang-Mills theories to first loop order found in Equation \ref{eq:betafunction YM 1loop}
\begin{equation}
    \beta_{g_{1-loop}}=\frac{-g^3}{16\pi^2}\left[ \frac{11}{3}C_2(H)-\frac{4}{3}n_fC(R)\right]\;,
\end{equation}
this is generic and without a specific fermion representation. First, we want to consider the beta function of the fine structure constant $\alpha_g=\frac{g^2}{16\pi^2}$ instead of the coupling itself
\begin{equation}
    \beta_{\alpha_g}=\frac{\text{d}\alpha_g}{\text{d}\ln \mu}=\frac{2g}{16\pi^2}\frac{\text{d}g}{\text{d}\ln \mu}\;.
\end{equation}
Next we can use that we are considering $\operatorname{SU}(N)$. This results in the Casimir operator $C_2(H)$ to be equal to $N$ \cite{Peskin:1995ev}.\comment{maybe do the calculation earlier}. Thus, the betafunction can be written as
\begin{equation}
    \beta_{\alpha_{g}}={-\alpha_g^2}\left[ \frac{11}{3}N-\frac{4}{3}n_fC(R)\right]\;,
\end{equation}
Next lets consider the second term which depends on the number of fermions for each representation. This is where chiral theories differ from vector-like theories. We consider the same fermionic representations as we had in the lagrangian in Equation \ref{eq:lagrangian YM Weyl}. To do so, it is necessary to specify the flavors of each fermionic representation. This depends on gauge anomalies, also called $\left[\operatorname{SU}(N) \right]^3$ anomalies \cite{Peskin:1995ev}. This anomaly appears in triangle diagrams of three gauge currents. For the theory to be valid the anomaly has to vanish. This happens in the SM. For $\operatorname{SU}(3)_c$ it is trivial because the theory is vector-like and thus have an equal amount of left- and righthanded fermions in the same representation. However, for $\operatorname{SU}(2)_L$ it is not as trivial because it is chiral. For three $\operatorname{SU}(2)_L$ gauge currents the anomaly cancels because the representation is pseudoreal. However, the anomaly can still happen between $\operatorname{SU}(2)_L$ gauge currents and $\operatorname{U}(1)$ or gravity. These combinations still cancel due to the definition of the hypercharge and \comment{why for gravity} \cite{Peskin:1995ev}. 
Thus, to ensure that our chiral theories are anomaly free, it is important that the different representations are build so their anomaly coefficients $A(r)$ cancel
\begin{equation}
    \sum_{fermions}A(r)=\mathcal{A}\overset{!}{=}0\;,
\end{equation}
For a symmetric representation $A(\Psi_{sym})=N+4$ and antisymmetric representation $A(\Psi_{Anti-sym})=N-4$. The fundamental and anti-fundamental representation is given in Equation \ref{eq:anomaly coefficient}. 
\begin{equation}
    \mathcal{A} = N\pm 4 + \bar p A(\bar{\square}) + p A({\square})= N\pm 4 - \bar p  +p= 0 \;,
\end{equation}
where $\bar p$ is the amount of Weyl fermions transforming in the anti-fundamental representation, that is the number of fermion flavors transforming under an anti-fundamental. $p$ is equivalent just for the fundamental representation. Thus, $\bar p$ is restricted to be equal to:
\begin{equation}
    \bar p= N\pm 4 + p  \;.
\end{equation}
Thus, when summing over the Dynkin index for all fermion representations, the anomaly free expression of $\bar p $ is used 
\begin{equation}
    n_f T(R) = T(\Psi) + \bar p T(\bar{\square}) + p T({\square})=\frac{1}{2}\left(N\pm2 + N\pm 4 +p +p\right)=N+p\pm 3\;,
\end{equation}
where $T(\square)=T(\bar \square)$ is given by equation \ref{eq: Dynkin index fund. rep} and $T(\Psi)=\frac{N\pm2}{2}$, where $+$ is for symmetric and $-$ is for anti-symmetric representations. \comment{where does this come from} 
Now inserting this in the beta function, remembering that $n_f$ is the number of Dirac fermions, so we need to divide that term by two, when inserting the amount of Weyl fermions 
\begin{equation}
    \beta_{\alpha_{g}}={-\alpha_g^2}\left[ \frac{22}{3}N-\frac{4}{3}N \mp \frac{12}{3}-\frac{4}{3}p\right]={-\alpha_g^2}\left[ \left(6-\frac{4}{3}x\right)N\mp 4\right]\;,
\end{equation}
where $x=p/N$. From this the fixed point can be found to one loop order by setting the beta function equal to zero as explained in \ref{sec: XX}, yielding two solutions
\begin{equation}
    \beta_{\alpha_{g}}=0\Rightarrow \alpha_g=0 \vee 6N-\frac{4}{3}xN \mp 4=0\;,
\end{equation}
where the first solution is the trivial UV fixed point, also called the gaussian fixed point. The second fixed point is for $x=x_{FP}=\frac{9}{2}\mp \frac{3}{N}$ \cite{M_lgaard_2017} \comment{need to do the fix point analysis - first the veneziano limit takes p and N to infinity but keep x=p/N constant, this makes the explicit chiral factor disappear. I think I need to do the change of variables also for vector like theories to really see the difference.}.

\subsection{Calculation of the effective potential}
\begin{equation}
    V_1(h, s) = \sum_i \pm n_i \frac{m_i^4(h, s)}{64\pi^2} \left( \log \left[ \frac{m_i^2(h, s)}{Q^2} \right] - C_i \right)
\end{equation}
and 
\begin{equation}
    V_{\text{eff}}(\varphi) = \frac{\lambda}{4!}\varphi^4 + \frac{\lambda^2 \varphi^4}{256\pi^2} \left( \log \frac{\varphi^2}{Q^2} - \frac{25}{6} \right) + O(\lambda^3)
\end{equation}

For $\ln 2/\lambda =8/3$ and $m^2=\lambda\phi^2/2$


\section{GG model with one scalar}
\begin{table}[h] 
    \centering 
    \begin{tabular}{|c|c|c|c|c|} \hline Fields & $[SU(N)]$ & $SU(N-4)$  & $U(1)$ \\ \hline $\Psi$ & $\square\square$ & 1 & $N-4$ \\ $\psi$ & $\bar\square$ & $\bar\square$ & $-(N-2)$ \\ \hline $\phi$ & $\bar\square$ & 1  & 2 \\ \hline 
    \end{tabular} 
    \caption{OBS squares need to be ontop of each other \\
    $\mathcal{L}= -\frac{1}{4} F_{\mu\nu}^a F^{\mu\nu\,a}+i \bar \Psi_{ij}\slashed{D}\Psi_{ij}+i \bar \psi_i^a\slashed{D}\psi_i^a+(D_\mu \phi^*D^\mu\phi)+(y^a\Psi_{ij} \psi ^a_{\, i} H^j + h.c.)+\lambda H^\dagger_i H_i H^\dagger_j H_j 
    \;.$} 
    \label{tab:GGp0} 
\end{table}
\begin{equation}
    \text{SU}(N)\times \text{SU}(N-4)\times \text{U}(1)
\end{equation}
where the transformation of the scalar, anti-fundamental fermion, and two index antisymmetric fermion under each group is denoted in Table \ref{}.
To write the Lagrangian down, we need to ensure that all terms will be invariant under $U(1)$ and that no indices remains open. To do this we start by writing down the kinetic terms for all particles. Let us first consider the free theory:
\begin{equation}
    \mathcal{L}_{free}= i \bar \Psi_{ij}\sigma^\mu \partial_\mu\Psi_{ij}+i \bar \psi_i^a\sigma^\mu \partial_\mu\psi_i^a+\partial_\mu (\phi^*)_i\partial^\mu\phi_i\;,
\end{equation}
where $\sigma^\mu$ is used because the fermions are chiral and thus are massless. Next, we want to consider the interacting theory, which we do by promoting the global $U(1)$ symmetry of the Lagrangian to a local symmetry. To ensure that ... we introduce the covariant derivative $\partial_\mu \to D_\mu= \partial_\mu - ie(T^a)^i_{\,j}A_\mu^a$
\begin{equation}
    \mathcal{L}_{kinetic}= -\frac{1}{4} F_{\mu\nu}^a F^{\mu\nu\,a}+i \bar \Psi_{ij}\slashed{D}\Psi_{ij}+i \bar \psi_i^a\slashed{D}\psi_i^a+(D_\mu \phi^*D^\mu\phi)\;,
\end{equation}
where $F_{\mu\nu}^a=\partial_\mu A_\nu^a-\partial_\nu A_\mu ^a+g f^{abc}A_\mu^bA_\nu^c$, with $A_\mu$ being the connection. This gives us our first coupling constant, $g$. If we write out the first term, we obtain
\begin{equation}
    F_{\mu\nu}^a F^{\mu\nu\,a}\approx(\partial_\mu A_\nu^a)^2+ g f^{abc}(\partial_\mu A_\nu^a)A_\mu^bA_\nu^c +g^2 f^{abc}f^{def}A_\mu^bA_\nu^cA_\mu^dA_\nu^e\;,
\end{equation}
which gives us the gauge boson propagator and the two gauge bosons vertices
\comment{feynman diagrams}
The three remaining kinetic terms gives us the scalar and two fermion propagators 
\comment{feynman diagram}
Next, we consider all the possible interaction terms, which we can write in the Lagrangian. Consider first, the possible Yukawa interactions. Terms like $\Psi_{ij}\bar A_{ij}H^i$ and $\psi \bar \psi H^i$ are not invariant under $U(1)$. However, mixing the two fermion representations, is possible, this however gives an open gauge index from the anti-fundamental fermion. To fix this we take the coupling with the same gauge index, giving us $2a=2(N-4)$ Yukawa terms including the hermitian conjugated terms
\begin{equation}
    \mathcal{L}_{Yukawa}= y^a\Psi_{ij} \psi ^a_{\, i} H^j + h.c.\; ,
\end{equation}
Lastly, we have to consider possible quartic scalar terms. Again, ensuring $U(1)$ invariant terms, we can write terms like
\begin{equation}
    \mathcal{L}_{quartic}= 
    \lambda_1 (H^\dagger_i H_j)^2 
    + \lambda_2 (\bar \Psi_{ij}\Psi_{ij})^2
    +\lambda_3 (\bar \psi_{i}\psi_{j})^2
    +\lambda_4 (H^\dagger_i H_j)(\bar \psi_{i}\psi_{j})
    +\lambda_5 (H^\dagger_i H_j)(\bar \Psi_{ij}\Psi_{ij})
    +\lambda_2 (\bar \psi_{i}\psi_{j})(\bar \Psi_{ij}\Psi_{ij})
    +\lambda_2 (\bar \Psi_{ij}\Psi_{ij})^2+...\;,
\end{equation}
and similar terms. However, we will impose another condition namely that the theory needs to be renomalizablel. This means that the coupling, shouldn't have negative dimensions. The quartic scalar lagrangian thus simplifies to one term
\begin{equation}
    \mathcal{L}_{quartic}= 
    \lambda H^\dagger_i H_i H^\dagger_j H_j \;,
\end{equation}
thus we also have one scalar coupling, with the coupling constant $\lambda$. From the scalar and the yukawa terms we obtain three new vertices. The full Lagrangian thus reads
\begin{equation}
    \mathcal{L}= -\frac{1}{4} F_{\mu\nu}^a F^{\mu\nu\,a}+i \bar \Psi_{ij}\slashed{D}\Psi_{ij}+i \bar \psi_i^a\slashed{D}\psi_i^a+(D_\mu \phi^*D^\mu\phi)^2+(y^a\Psi_{ij} \psi ^a_{\, i} H^j + h.c.)+\lambda H^\dagger_i H_i H^\dagger_j H_j 
    \;.
\end{equation}

\subsection{Beta functions of the Lagrangian}
To analyze wether the theory is asymptotical free, we will use the Callan-Symanzik Equation \ref{h}. The beta function will first be computed for the gauge coupling
\begin{equation}
    \beta(\alpha_g)\mu\frac{\text{d}\alpha_g}{\text{d}\ln\mu}= b_0 \alpha_g^2\;,
\end{equation}
where $\alpha_g=\frac{g^2}{4\pi}$ and $\mu$ is an arbitrary energy-scale. The beta equation for the gauge interaction can be computed by calculating the two, one-loop diagrams\comment{insert feynman diagrams that contribute}. To compute this we first need to renormalize the Lagrangian.  

For the gauge coupling the betafunction is thus (without scalars)
\begin{equation}
    \beta(g)=-\frac{g^3}{(4\pi)^2}\left[\frac{11}{3}C_2(G)-\frac{2}{6}T(R)\right]\;,
\end{equation}
This took quite a lot of computation, and I realized so would the remaning beta functions. I then chose to use a numerical approach, the beta function with scalars was then looked up:
\begin{equation}
    \beta(g)=-\frac{g^3}{(4\pi)^2}\left[\frac{11}{3}C_2(G)-\frac{2}{6}T(F)-\frac{1}{3}T(S)\right]\;,
\end{equation}
\subsection{BY model with p=0 and one scalar}
\begin{table}[h] 
    \centering 
    \begin{tabular}{|c|c|c|c|c|} \hline Fields & $[SU(N)]$ & $SU(N-4)$  & $U(1)$ \\ \hline $\chi$ & $\square\square$ & 1 & $N+4$ \\ $\psi$ & $\bar\square$ & $\bar\square$ & $-(N+2)$ \\ \hline $\phi$ & $\bar\square$ & 1  & -2 \\ \hline 
    \end{tabular} 
    \caption{ caption} 
    \label{tab:BYp0} 
\end{table}
Considering a chiral model with scalars, we now focus on the Bars-Yankielowicz model. As seen in table \ref{tab:BYp0}, the main difference from the GG model is that instead og having a two index antisymmetric representation, we have a two-index symmetric representation. That is the fermionic field $\chi_{ij}=\chi_{ji}$. The lagrangian is similar to the one of GG and can be expressed as
 \begin{align*}\mathcal{L}= -\frac{1}{4} F_{\mu\nu}^a F^{\mu\nu\,a}+i \bar \chi^{ij}\slashed{D}\chi_{ij}+i \bar \psi_{i\,a}\slashed{D}\psi_i^a+(D_\mu \phi^{*\,i})(D^\mu\phi_i)+(y^a\chi^{ij} \psi ^a_{\, i} \phi_j + h.c.)+\lambda (\phi^{\dagger}_i \phi^i)( \phi^{\dagger}_j \phi^j )
    \;.
 \end{align*}
There is another scalar quartic term however for (anti-)fundamental representations it is not linearly independent on the quartic scalar term in the lagrangian. Let us start by proving that 
\begin{align}
    (\phi_i^\dagger T_i ^{\,A\,j}\phi^j)(\phi_k^\dagger T_k ^{\,A\,l}\phi^l)\;,
\end{align}
where the generators $T^A$ upholds the Fierz identity
\begin{align}
    T_i ^{\,A\,j}T_k ^{\,A\,l}=\frac{1}{2}\left(\delta_i^{\,l}\delta_k^{\,j}-\frac{1}{N}\delta_i^{\,j}\delta_k^{\,l}\right) \;.
\end{align}
The Fierz identity can be shown to be true for fundamental representations of $SU(N)$ by considering \dots
The generators of $SU(N)$ can be written as $n\times n $ traceless Hermitian matricies $T^a$. The noramlisation of theses matricies is first defined to be 
\begin{equation}
      \text{tr}(T^a T^b)=\frac{1}{2}\delta^{ab}\;,
\end{equation}
using this we can write a general matrix $M$ as a linear combination of the identity matrix and the generators of the Lie algebra
\begin{equation}
      M=n_0 I+n_aT^a\;,
\end{equation}
where $n_0$ and $n_a$ are coefficients. Now let us take the trace on both sides and simplify using that $\tr(T^a)=0$ and $\tr(I)=N$
\begin{equation}
      \tr(M)=n_0 \tr(I)+n_a\tr(T^a)=n_0N\;.\label{eq:tr1}
\end{equation}
Next using the definition of the normalisation of the generators
\begin{equation}
      \tr(MT^b)=n_0 \tr(IT^b)+n_a\tr(T^aT^b)=\frac{n_b}{2}\;.\label{eq:tr2}
\end{equation}
Now using Equation \ref{eq:tr1,eq:tr2}, the constants $n_0,n_a$ can be replaced, and the matrix $M$ can be written as
\begin{equation}
      M=\frac{\tr(M)}{N} I+2\tr(MT^a)T^a\;,
\end{equation}
the matrix decomposition can be written in index notation
\begin{equation}
      M_{ij}=\frac{\tr(M)}{N} \delta_{ij}+2\tr(MT^a)T^a_{ij}\;,
\end{equation}
Using that it is in Euclidian space with the metric $\delta_{ij}$, the traces cab be simplified, and the lhs can be rewritten
\begin{equation}
      \delta_{jk}\delta_{il}M_{lk}=\frac{1}{N} \delta_{kl}M_{lk}\delta_{ij}+2\delta_{ln}M_{lk}T^a_{kn}T^a_{ij}\;,
\end{equation}
which can be simplified to 
\begin{equation}
      \delta_{jk}\delta_{il}=\frac{1}{N} \delta_{kl}\delta_{ij}+2T^a_{kl}T^a_{ij}\;,
\end{equation}
Isolating 
\begin{equation}
      T^a_{kl}T^a_{ij}=\frac{1}{2}\left(\delta_{jk}\delta_{il}-\frac{1}{N} \delta_{kl}\delta_{ij}\right)
\end{equation}
Inserting this back into the term gives us
\begin{align}
    \frac{1}{2} (\phi_i^\dagger \phi^j)(\phi_k^\dagger \phi^l)\left(\delta_i^{\,l}\delta_k^{\,j}-\frac{1}{N}\delta_i^{\,j}\delta_k^{\,l}\right)=\frac{1}{2}\left[\phi^{\dagger}_i \phi^j \phi^{\dagger}_j \phi^i-\frac{1}{N}\phi^{\dagger}_i \phi^i \phi^{\dagger}_k \phi^ik\right]=\frac{1}{2}\left(1-\frac{1}{N}\right) (\phi^{\dagger}_i \phi^i))^2\;,
\end{align}
so it is not linarly independt on the quartic term in the lagrangian. 
\subsection{Beta functions}
For the couplings defined as
\begin{align*}
    \alpha_g = \frac{g^2}{(4\pi)^2}, \quad \alpha_y = \frac{y^2}{(4\pi)^2}, \quad\alpha_\lambda = \frac{\lambda}{(4\pi)^2}
\end{align*}
For GG model:
\begin{align}
    \beta_{\alpha_g}&=-\frac{11+18N}{3}\alpha_g^2\\
    \beta_{\alpha_y}&= \frac{\alpha_y }{2}\left[(3 N-1)
   \alpha_y+6 \alpha_g \left(-3 N+2+\frac 5N\right)\right]\\
    \beta_{\alpha_\lambda}&=\alpha_\lambda\left[2(N+1)\alpha_y-\frac{6 \alpha_g 
   \left(N^2-1\right)}{N}+4 \alpha_\lambda (N+4)\right]-\frac{1}{2} (N+3)
  \alpha_y^2+\frac{3 \alpha_g^2 \left(N^3+N^2-4 N+2\right)}{4 N^2}
\end{align}
For BY model:
\begin{align}
    \beta_{\alpha_g}&=-(1+6N)\alpha_g^2\\
    \beta_{\alpha_y}&= \frac{\alpha_y}{2}\left[(3 N+5)\alpha_y+6 \alpha_g\left(-3 N-2+\frac 5N\right) \right]
   \\
   \beta_{\alpha_\lambda}&=\alpha_\lambda\left[2(N+1)\alpha_y-\frac{6 \alpha_g 
   \left(N^2-1\right)}{N}+4 \alpha_\lambda (N+4)\right]-\frac{1}{2} (N+3)
  \alpha_y^2+\frac{3 \alpha_g^2 \left(N^3+N^2-4 N+2\right)}{4 N^2}
\end{align}
For large N both models reduce to
\begin{align}
    \beta_{\alpha_g}&\approx-6N\alpha_g^2\\
    \beta_{\alpha_y}&\approx N\alpha_y\left(-9\alpha_g+\frac{3}{2}\alpha_y\right)\\
    \beta_{\alpha_\lambda}&\approx N\left(\alpha_\lambda\left[2\alpha_y-6 \alpha_g 
   +4 \alpha_\lambda \right]-\frac{1}{2}
  \alpha_y^2+\frac{3}{4} \alpha_g^2 \right)
\end{align}
\subsection{CAF}
After the fixed point analysis we have the following conditions when the Yukawa and gauge coupling is not on fixed flow:
\begin{equation}
    b_0<0,\quad\quad b_0-c_1>0,\quad\quad (b_0-d_2)^2-4d_1d_4\geq 0,\quad\quad b_0-d_2+\sqrt{(b_0-d_2)^2-4d_1d_4}>0
\end{equation}
